\chapter*{Tóm tắt}
\label{summary}

Trong nhiều năm trở lại đây, các mô hình mạng nơ-ron nhiều tầng ẩn đã tạo nên những bước cải tiến lớn trong vô số lĩnh vực khoa học khác nhau. Để đạt được những kết quả đó, tất cả mô hình đều yêu cầu một quá trình điều chỉnh bộ trọng số để mô hình có thể thích ứng với dữ liệu huấn luyện và dự đoán trên dữ liệu mới một cách chính xác. Vì vậy, một phương pháp đi tìm bộ trọng số hiệu quả sẽ cải thiện độ chính xác của mô hình, rút ngắn thời gian huấn luyện, từ đó nâng cao tính ứng dụng của các mô hình mạng nơ-ron nhiều tầng ẩn.

Khóa luận tập trung tìm hiểu về những khó khăn trong việc huấn luyện mạng nơ-ron nhiều tầng ẩn và cách mà các thuật toán tối ưu giải quyết những khó khăn đó, với trọng tâm là thuật toán Adam. Thuật toán Adam được giới thiệu trong bài báo "Adam: A Method for Stochastic Optimization" công bố tại hội nghị "International Conference on Learning Representation 2015". Ý tưởng của thuật toán là kết hợp hai hướng tiếp cận: (1) sử dụng quán tính để tăng tốc và giảm dao động, và (2) thích ứng lượng cập nhật cho từng trọng số; với mục tiêu kế thừa được những điểm mạnh và đồng thời cải thiện những điểm yếu của mỗi hướng tiếp cận.

Kết quả đạt được của khóa luận là cài đặt lại thuật toán Adam cùng các thuật toán liên quan và tái tạo được một phần kết quả bài báo gốc. Khóa luận cũng thực hiện thêm các thí nghiệm nhằm phân tích và làm rõ cách mỗi phương pháp giải quyết các khó khăn trong việc huấn luyện mạng nơ-ron nhiều tầng ẩn.