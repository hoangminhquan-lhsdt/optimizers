\chapter{Các kết quả thí nghiệm}
\label{Chapter4}

\textit{Trong chương này, chúng tôi trình bày các kết quả thí nghiệm nhằm đánh giá những nội dung đã trình bày ở chương 3. Cho các thí nghiệm về nguyên lý, chúng tôi sử dụng dữ liệu được tạo ngẫu nhiên và hàm lỗi Mean Squared Error; với các thí nghiệm thực tế, chúng tôi sử dụng bộ dữ liệu MNIST và CIFAR10 và hàm lỗi Cross Entropy. Kết quả thí nghiệm cho thấy thuật toán mà chúng tôi cài đặt có thể xấp xỉ kết quả mà bài báo công bố, tuy nhiên chúng tôi nhận thấy rằng bộ siêu tham số được sử dụng để tái tạo kết quả có thể không cho kết quả tốt nhất cho tất cả thuật toán. Ngoài ra, các kết quả thí nghiệm cũng cho thấy các trường hợp mà các thuật toán khác gặp khó khăn, và cách mà Adam vượt qua các khó khăn đó. Cuối cùng, các thí nghiệm cho thấy tốc độ tối ưu hóa của các thuật toán trên các mô hình mạng nơ-ron thực tế với nhiều tầng ẩn gồm các cấu trúc khác nhau.}

\section{Các thiết lập thí nghiệm}

Chúng tôi sử dụng ngôn ngữ lập trình Python và thư viện Numpy cho các thí nghiệm nguyên lý, thư viện Pytorch cho các thí nghiệm thực tế. Cả thư viện Numpy và Pytorch đều cung cấp khả năng tăng tốc thực thi bằng việc véc-tơ hóa tính toán trên nền C/C++. Thư viện Numpy tập trung vào các chức năng tính toán đại số trên CPU, phù hợp với các thí nghiệm đơn giản; trong khi thư viện Pytorch là một thư viện máy học có thể xây dựng các mạng nơ-ron nhiều tầng ẩn phức tạp cùng với khả năng thực thi song song trên GPU. Loại GPU mà chúng tôi sử dụng là NVIDIA RTX 2080.

Chúng tôi sử dụng ngôn ngữ lập trình Python và thư viện Numpy cho các thí nghiệm nguyên lý, thư viện Pytorch cho các thí nghiệm thực tế. Cả thư viện Numpy và Pytorch đều cung cấp khả năng tăng tốc thực thi bằng việc véc-tơ hóa tính toán trên nền C/C++. Thư viện Numpy tập trung vào các chức năng tính toán đại số trên CPU, phù hợp với các thí nghiệm đơn giản; trong khi thư viện Pytorch là một thư viện máy học có thể xây dựng các mạng nơ-ron nhiều tầng ẩn phức tạp cùng với khả năng thực thi song song trên GPU. Loại GPU mà chúng tôi sử dụng là NVIDIA RTX 2080, ngoài ra chúng tôi cũng sử dụng NVIDIA Tesla T4 trên nền tảng Google Colaboratory.

\section{Các kết quả thí nghiệm}

\subsection{Kết quả của thuật toán cài đặt so với bài báo}

Placeholder

\subsection{Phân tích trường hợp bề mặt lỗi align và không align với tham số}

Placeholder

\subsection{Vấn đề dữ liệu thưa và nhiều lỗi}

Placeholder

\subsection{Thử nghiệm trên mô hình VGG16}

Placeholder

\subsection{Thử nghiệm trên mô hình RNN}

Placeholder